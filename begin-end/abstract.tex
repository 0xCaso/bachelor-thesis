% !TEX encoding = UTF-8
% !TEX TS-program = pdflatex
% !TEX root = ../thesis.tex

%**************************************************************
% Sommario
%**************************************************************
\cleardoublepage
\phantomsection
\pdfbookmark{Summary}{Summary}
\begingroup
\let\clearpage\relax
\let\cleardoublepage\relax
\let\cleardoublepage\relax

\chapter*{Abstract}

Nowadays, most of our data is owned by private companies, and everyone knows 
everything about us because privacy online is not well preserved. Imagining a 
world different from this is difficult, but things can change thanks to 
\acrfull{ssi}. SSI approach aims to bring credentials back to the 
actual owners, the people. This is possible through cryptography and secure 
authentication layers (e.g., OAuth, OpenIDConnect). The developed product embraces 
this philosophy and offers a solution where the users are the holders, issuers, or 
verifiers of Verifiable Credentials (VCs). Specifically, will be developed software 
agents who create, issue, verify, modify or even revoke the credentials, leveraging 
an SSI Kit.\\\\
In this thesis, we propose a methodology to merge SSI off-chain (i.e., outside the 
blockchain) operations with on-chain smart contracts. In particular, the job has been 
divided into three macro stages: firstly, has been done a deep dive into the SSI 
technology, studying all of its primitives and analyzing the problem; secondly, has 
been developed a \acrfull{sdk}, which enabled us to dialog with an 
SSI Kit (off-chain logic); in the meantime, my friend and co-worker Matteo Midena 
developed the smart contracts (on-chain logic); finally, off-chain and on-chain 
solutions has been merged in a proof of concept web application.
One of the final features is that the verifier (who inspects the validity of the 
credentials) can reflect on-chain the off-chain verification results, saving time 
for the following examinations. Improvements and additional features are needed to 
complete the software, but this constitutes a good baseline for future works.
\endgroup			

\vfill

