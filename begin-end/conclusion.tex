\chapter*{Conclusion}
Taking up what was said in the abstract, we can confirm that the final product embraces 
SSI concepts and tries to take it to the next level with the help of smart contracts.  
The developed SDK enables the issuers to release verifiable credentials to holders who 
own them and present them to verifiers who can confirm their validity. Thanks to smart 
contracts, we can register on-chain verification results to make them public and speed 
up the following verifications. The final proof of concept proves that off-chain SSI 
primitives can be reflected on-chain. To do so, compromises are needed to preserve 
privacy (e.g., using a permissioned blockchain as VDR simplifies things).
\vspace{0.3cm}\\
Our final product leaves room for numerous additional features, meaning that this was 
not thought of as definitive software but as a beginning for the following implementations.
We are confident that Self-Sovereign Identity will catch on sooner or later, and the 
conclusive result of this thesis offers just a taste of what these innovative and
promising technologies could bring.