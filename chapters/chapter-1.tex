% !TEX encoding = UTF-8
% !TEX TS-program = pdflatex
% !TEX root = ../thesis.tex

\chapter{Introduction}
This chapter will introduce the problem: what we are analyzing, why 
this problem exists, how it is defined, and how it can be resolved. 
Also will be introduced the company, the internship, and the work 
methodology.
% /*//////////////////////////////////////////////////////////////
%                             SECTION
% //////////////////////////////////////////////////////////////*/
\section{The problem}
As stated in the abstract, the main problem is preserving the ownership 
of people's data. In order to achieve this objective, we will pass through 
problems like interoperability, privacy safeguarding, law compliance, security,
and others.
\vspace*{0.3cm}\\
Assuming we can create a system where people hold their credentials 
(called Verifiable Credentials, or VCs):
\begin{enumerate}
    \setlength\itemsep{-0.3em}
    \item How can these credentials be shown to and verified in the 
    same manner by different actors?
    \item Can we demonstrate something without revealing it, preserving
    our privacy this way?
    \item Is it possible to save on blockchain people's data, or are we
    going against specific privacy laws?
    \item Are credentials susceptible to attacks from hackers trying to 
    steal our data?
\end{enumerate}
Thanks to Self-Sovereign Identity, we can give a positive answer to all
of these questions, but as is often the case, we have to deal with 
compromises.

\paragraph{Worldwide scenario}
The current situation is clear. Every time we interact with a new website,
we may want to interact with it, and to do so, we have to register to create 
a profile. In this phase, we have to give our data to the company, and they 
will be stored in their databases.

\paragraph{Problem identification}
Let us now try to answer the previous questions to check how the present 
context is managed:
\begin{enumerate}
    \item \textbf{Interoperability} | We could have two cases. In the first one, 
    we use a technology that enables us to use our existing account on 
    multiple websites, which integrates this solution, for example, "Sign-Up 
    with Google". Here our data is owned by Google, which shares them (if we 
    grant permission) with third parties, and no one prohibits third parties 
    from keeping our shared data saved. In the second case, we must register 
    each time if the third party does not integrate other "Sign-Up with \textbf{*}" 
    solutions. In both cases, third parties can collect our data (in the 
    first case, Google explicitly knows our interests, but there is a minimum
    degree of interoperability). Also, in most cases, companies will let us 
    create multiple accounts without verifying our data (one exception to 
    this is the use of KYC).
    \item \textbf{Privacy} | In some cases, we must show our data
    with complete transparency: for example, the police stop us on the 
    street and ask for our details. Nevertheless, let us suppose we want to 
    demonstrate something without revealing the details. For example, 
    someone has graduated and wants to demonstrate it without revealing 
    his final grade. We can do this thanks to a cryptography method 
    called \textit{Zero-Knowledge Proof}. However, this has not yet been
    implemented in most current systems.
    \item \textbf{Law compliance} | If we consider saving users' data in 
    blockchains, this problem does not exist as we examine centralized 
    systems which do not use them. By the way, of course, there are privacy 
    laws companies must follow (like GDPR).
    \item \textbf{Security} | Our information is stored in databases. With 
    a data breach, considering a centralized system, a malicious actor can 
    access all users' data at once. Sadly, this happens often. So often 
    that someone has made a website where anyone can check if his data has 
    been stolen online at least once (\href{https://haveibeenpwned.com/}
    {Have I Been Pwned?}).
\end{enumerate}

\paragraph{Problem statement}
With the above considerations, it is clear that the existing systems work 
but could be significantly improved. In fact, interoperability enhancement 
would mean privacy and security penalization. Compromises exist, but if the 
system is well designed, they can be significantly reduced or at least 
moved to less dangerous areas. Here, the need for a more secure way to store
user data arises. A way that intersects the analyzed points, bringing new 
power to people and reducing that of companies. This is the Self-Sovereign 
Identity's principle, which the developed solution will leverage.

\paragraph{Approaches} SSI's concept is pretty simple, as opposed to its 
(in development) implementation. Everyone has different relationships or 
unique sets of identifying information. This information could include 
birth date, citizenship, university degrees, or business licenses. In the 
physical world, these are represented as cards and certificates that the 
identity holder holds in their wallet or a safe place like a safety deposit 
box. They are presented when the person needs to prove their identity or 
something about it.\\
Self-sovereign identity (SSI) brings the same freedom and personal autonomy
to the internet in a safe and trustworthy identity management system. 
SSI means the individual (or organization) manages the elements that make 
up their identity, and he digitally controls access to those credentials,
called Verifiable Credentials (or VCs). They are digital representations of
information that can be verified by a third party.
\vspace*{0.3cm}\\
This is achievable by involving three participants:
\begin{enumerate}
    \item \textbf{Holder} | The holder is an individual in the scenario, 
    although it can also be an organization/company. The holder is the 
    entity that holds the credential.
    \item \textbf{Issuer} | The issuer is the institution, be it a company, 
    certifier body, or governmental organization, that has been awarded a 
    level of trust to provide information (i.e., a public body that issued 
    a passport)
    \item \textbf{Verifier} | The verifier is the individual, organization,
    company, or government with whom the holder must prove information's 
    legitimacy and trustworthiness.
\end{enumerate}
The Verifiable Data Registry grants the trust: here are stored schemas and 
identifiers (linked to the credentials) that the verifiers use to check 
data validity without the issuer's intervention.
\vspace*{0.3cm}\\
To make a preliminary check of this solution's viability, let us try to 
answer the previous four questions, considering the new scenario:
\begin{enumerate}
    \item \textbf{Interoperability} | 
    \item \textbf{Privacy} | 
    \item \textbf{Law compliance} | 
    \item \textbf{Security} | 
\end{enumerate}
Check delle 4 domande --> sono ok --> 10 principi
% /*//////////////////////////////////////////////////////////////
%                             SECTION
% //////////////////////////////////////////////////////////////*/
\section{Basic use cases}
% /*//////////////////////////////////////////////////////////////
%                             SECTION
% //////////////////////////////////////////////////////////////*/
\section{Company, internship, work methodology}
