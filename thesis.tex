        %%******************************************%%
        %%                                          %%
        %%        Modello di tesi di laurea         %%
        %%            di Andrea Giraldin            %%
        %%                                          %%
        %%             2 novembre 2012              %%
        %%                                          %%
        %%******************************************%%


% I seguenti commenti speciali impostano:
% 1. 
% 2. PDFLaTeX come motore di composizione;
% 3. thesis.tex come documento principale;
% 4. il controllo ortografico italiano per l'editor.

% !TEX encoding = UTF-8
% !TEX TS-program = pdflatex
% !TEX root = thesis.tex
% !TEX spellcheck = it-IT

% PDF/A filecontents
\RequirePackage{filecontents}
\begin{filecontents*}{\jobname.xmpdata}
  \Title{Document’s title}
  \Author{Author’s name}
  \Language{it-IT}
  \Subject{The abstract, or short description.}
  \Keywords{keyword1\sep keyword2\sep keyword3}
\end{filecontents*}

\documentclass[10pt,                    % corpo del font principale
               a4paper,                 % carta A4
               oneside,                 % impagina per fronte-retro
               openright,               % inizio chapters a destra
               english,                 
               ]{book}    

%**************************************************************
% Importazione package
%************************************************************** 

\PassOptionsToPackage{table, dvipsnames}{xcolor} % colori PDF/A

\usepackage{colorprofiles}

\usepackage[a-2b,mathxmp]{pdfx}[2018/12/22]
                                        % configurazione PDF/A
                                        % validare in https://www.pdf-online.com/osa/validate.aspx

%\usepackage{amsmath,amssymb,amsthm}    % matematica

\usepackage[T1]{fontenc}                % codifica dei font:
                                        % NOTA BENE! richiede una distribuzione *completa* di LaTeX

\usepackage[utf8]{inputenc}             % codifica di input; anche [latin1] va bene
                                        % NOTA BENE! va accordata con le preferenze dell'editor

\usepackage[english]{babel}    % per scrivere in italiano e in inglese;
                                        % l'ultima lingua (l'italiano) risulta predefinita

\usepackage{bookmark}                   % segnalibri

\usepackage{caption}                    % didascalie

\usepackage{chngpage,calc}              % centra il frontespizio

\usepackage{csquotes}                   % gestisce automaticamente i caratteri (")

\usepackage{emptypage}                  % pagine vuote senza testatina e piede di pagina

\usepackage{epigraph}			% per epigrafi

\usepackage{eurosym}                    % simbolo dell'euro

%\usepackage{indentfirst}               % rientra il primo paragrafo di ogni sezione

\usepackage{graphicx}                   % immagini

\usepackage{hyperref}                   % collegamenti ipertestuali

\usepackage[binding=5mm]{layaureo}      % margini ottimizzati per l'A4; rilegatura di 5 mm

\usepackage{listings}                   % codici

\usepackage{microtype}                  % microtipografia

\usepackage{mparhack,fixltx2e,relsize}  % finezze tipografiche

\usepackage{nameref}                    % visualizza nome dei riferimenti                                      
\usepackage[font=small]{quoting}        % citazioni

\usepackage{subfig}                     % sottofigure, sottotabelle

\usepackage[italian]{varioref}          % riferimenti completi della pagina

\usepackage{booktabs}                   % tabelle                                       
\usepackage{tabularx}                   % tabelle di larghezza prefissata                                    
\usepackage{longtable}                  % tabelle su più pagine                                        
\usepackage{ltxtable}                   % tabelle su più pagine e adattabili in larghezza

\usepackage[toc, acronym]{glossaries}   % glossario
                                        % per includerlo nel documento bisogna:
                                        % 1. compilare una prima volta thesis.tex;
                                        % 2. eseguire: makeindex -s thesis.ist -t thesis.glg -o thesis.gls thesis.glo
                                        % 3. eseguire: makeindex -s thesis.ist -t thesis.alg -o thesis.acr thesis.acn
                                        % 4. compilare due volte thesis.tex.

\usepackage[backend=biber,style=ieee,hyperref,backref]{biblatex}
                                        % eccellente pacchetto per la bibliografia; 
                                        % produce uno stile di citazione autore-anno; 
                                        % lo stile "numeric-comp" produce riferimenti numerici
                                        % per includerlo nel documento bisogna:
                                        % 1. compilare una prima volta thesis.tex;
                                        % 2. eseguire: biber tesi
                                        % 3. compilare ancora thesis.tex.
\usepackage[bottom]{footmisc}           % footnote
\usepackage{tablefootnote}              % footnote for tables
\usepackage{mdwlist}                    
\usepackage{seqsplit}                   % split long sequences
\usepackage[many]{tcolorbox}
\usepackage{xpatch}

%**************************************************************
% file contenente le impostazioni della thesis
%**************************************************************

%**************************************************************
% Frontespizio
%**************************************************************

% Autore
\newcommand{\myName}{Matteo Casonato}                                    
\newcommand{\myTitle}{Owning your data through Self-Sovereign Identity: agents implementation for Verifiable Credentials interaction}

% Tipo di thesis                   
\newcommand{\myDegree}{Bachelor thesis}

% Università             
\newcommand{\myUni}{University of Padua}

% Facoltà       
\newcommand{\myFaculty}{Bachelor's Degree in Computer Science}

% Dipartimento
\newcommand{\myDepartment}{Department of Mathematics "Tullio Levi-Civita"}

% Titolo del relatore
\newcommand{\profTitle}{Prof. }

% Relatore
\newcommand{\myProf}{Alessandro Brighente}

% Luogo
\newcommand{\myLocation}{Padova}

% Anno accademico
\newcommand{\myAA}{2021-2022}

% Data discussione
\newcommand{\myTime}{September 2022}


%**************************************************************
% Impostazioni di impaginazione
% see: http://wwwcdf.pd.infn.it/AppuntiLinux/a2547.htm
%**************************************************************

\setlength{\parindent}{14pt}   % larghezza rientro della prima riga
\setlength{\parskip}{0pt}   % distanza tra i paragrafi

% PER STAMPARE TESI, COMMENTARE LE 4 RIGHE SOTTO
% E SETTARE TWOSIDE IN DOCUMENTCLASS

\pdfpageheight=\paperheight
\pdfpagewidth=\paperwidth
\setlength\oddsidemargin{\dimexpr(\paperwidth-\textwidth)/2 - 1in\relax}
\setlength\evensidemargin{\oddsidemargin}

%**************************************************************
% Impostazioni di biblatex
%**************************************************************
\bibliography{bibliography} % database di biblatex 

\defbibheading{bibliography} {
    \cleardoublepage
    \phantomsection 
    \addcontentsline{toc}{chapter}{\bibname}
    \chapter*{\bibname\markboth{\bibname}{\bibname}}
}

\setlength\bibitemsep{1.5\itemsep} % spazio tra entry
\setlength{\skip\footins}{1.2pc plus 10pt minus 2pt} % spazio tra entry e footnote

\DeclareBibliographyCategory{opere}
\DeclareBibliographyCategory{web}
\DeclareBibliographyDriver{online}{%
  \usebibmacro{bibindex}%
  \usebibmacro{begentry}%
  \usebibmacro{author/editor+others/translator+others}%
  \setunit{\adddot\addspace}%
    \iffieldundef{year}
        {}
        {\printtext[parens]{\usebibmacro{date}}}
  \setunit{\adddot\addspace}%
  \usebibmacro{title}%
  \setunit{\adddot\addspace}%
  \printlist{language}%
  \setunit{\adddot\addspace}%
  \usebibmacro{byauthor}%
  \setunit{\adddot\addspace}%
  \usebibmacro{byeditor+others}%
  \setunit{\adddot\addspace}%
  \printfield{version}%
  \setunit{\adddot\addspace}%
  \printfield{note}%
  \newunit\newblock
  \printlist{organization}%
  \setunit{\adddot\addspace}%
  \iftoggle{bbx:eprint}
    {\usebibmacro{eprint}}
    {}%
  \newunit\newblock
  \usebibmacro{url+urldate}%
  \setunit{\adddot\addspace}%
  \usebibmacro{addendum+pubstate}%
  \setunit{\bibpagerefpunct}\newblock
  \usebibmacro{pageref}%
  \newunit\newblock
  \usebibmacro{related}%
  \usebibmacro{finentry}%
}

\addtocategory{opere}{article:zkkyc}
\addtocategory{opere}{article:soulboundpaper}
\addtocategory{opere}{article:soulbound}
\addtocategory{web}{site:agile-manifesto}

\defbibheading{opere}{\section*{Bibliographical references}}
\defbibheading{web}{\section*{Websites consulted}}
\defbibheading{booklets}{\section*{Booklets / Presentations}}


%**************************************************************
% Impostazioni di caption
%**************************************************************
\captionsetup{
    tableposition=top,
    figureposition=bottom,
    font=small,
    format=hang,
    labelfont=bf
}

%**************************************************************
% Impostazioni di glossaries
%**************************************************************
\makeglossaries
%**************************************************************
% Acronimi
%**************************************************************
\newacronym{api}{API}{Application Programming Interface}
\newacronym{cli}{CLI}{Command Line Interface}
\newacronym{crud}{CRUD}{Create-Read-Update-Delete}
\newacronym{did}{DID}{Decentralized Identifier}
\newacronym{ebsi}{EBSI}{European Blockchain Services Infrastructure}
\newacronym{erc}{ERC}{Ethereum Request for Comments}
\newacronym{evm}{EVM}{Ethereum Virtual Machine}
\newacronym{gdpr}{GDPR}{General Data Protection Regulation}
\newacronym{json}{JSON}{JavaScript Object Notation}
\newacronym{jvm}{JVM}{Java Virtual Machine}
\newacronym{jwk}{JWK}{JSON Web Key}
\newacronym{jws}{JWS}{JSON Web Signature}
\newacronym{jwt}{JWT}{JSON Web Token}
\newacronym{kyc}{KYC}{Know Your Customer}
\newacronym{nft}{NFT}{Non-Fungible Token}
\newacronym{oidc}{OIDC}{OpenID Connect}
\newacronym{pem}{PEM}{Privacy Enhanced Mail}
\newacronym{poc}{POC}{Proof of Concept}
\newacronym{rest}{REST}{Representational State Transfer}
\newacronym{rfc}{RFC}{Request for Comments}
\newacronym{ui}{UI}{User Interface}
\newacronym{uuid}{UUID}{Universally Unique IDentifier}
\newacronym{ux}{UX}{User Experience}
\newacronym{vc}{VC}{Verifiable Credential}
\newacronym{vp}{VP}{Verifiable Presentation}
\newacronym{vdr}{VDR}{Verifiable Data Registry}
\newacronym{sdk}{SDK}{Software Development Kit}
\newacronym{ssi}{SSI}{Self-Sovereign Identity}
\newacronym{uri}{URI}{Uniform Resource Identifier}
\newacronym{w3c}{W3C}{World Wide Web Consortium}
\newacronym{zkp}{ZKP}{Zero-Knowledge Proof}
\newacronym{zksnark}{ZK-SNARK}{Zero-Knowledge Succinct Non-interactive ARguments of Knowledge}
 % database di termini


%**************************************************************
% Impostazioni di graphicx
%**************************************************************
\graphicspath{{img/}} % cartella dove sono riposte le immagini


%**************************************************************
% Impostazioni di hyperref
%**************************************************************
\hypersetup{
    %hyperfootnotes=false,
    %pdfpagelabels,
    %draft,	% = elimina tutti i link (utile per stampe in bianco e nero)
    colorlinks=true,
    linktocpage=true,
    pdfstartpage=1,
    pdfstartview=,
    % decommenta la riga seguente per avere link in nero (per esempio per la stampa in bianco e nero)
    %colorlinks=false, linktocpage=false, pdfborder={0 0 0}, pdfstartpage=1, pdfstartview=FitV,
    breaklinks=true,
    pdfpagemode=UseNone,
    pageanchor=true,
    pdfpagemode=UseOutlines,
    plainpages=false,
    bookmarksnumbered,
    bookmarksopen=true,
    bookmarksopenlevel=1,
    hypertexnames=true,
    pdfhighlight=/O,
    %nesting=true,
    %frenchlinks,
    urlcolor=webbrown,
    linkcolor=RoyalBlue,
    citecolor=webgreen,
    %pagecolor=RoyalBlue,
    %urlcolor=Black, linkcolor=Black, citecolor=Black, %pagecolor=Black,
    pdftitle={\myTitle},
    pdfauthor={\textcopyright\ \myName, \myUni, \myFaculty},
    pdfsubject={},
    pdfkeywords={},
    pdfcreator={pdfLaTeX},
    pdfproducer={LaTeX}
}

%**************************************************************
% Impostazioni di itemize
%**************************************************************
\renewcommand{\labelitemi}{$\ast$}
% \renewcommand{\labelitemi}{$\bullet$}
% \renewcommand{\labelitemii}{$\cdot$}
% \renewcommand{\labelitemiii}{$\diamond$}
%\renewcommand{\labelitemiv}{$\ast$}


%**************************************************************
% Impostazioni di listings
%**************************************************************
\lstset{
    language=[LaTeX]Tex,%C++,
    keywordstyle=\color{RoyalBlue}, %\bfseries,
    basicstyle=\small\ttfamily,
    %identifierstyle=\color{NavyBlue},
    commentstyle=\color{Green}\ttfamily,
    stringstyle=\rmfamily,
    numbers=none, %left,%
    numberstyle=\scriptsize, %\tiny
    stepnumber=5,
    numbersep=8pt,
    showstringspaces=false,
    breaklines=true,
    frameround=ftff,
    frame=single
} 


%**************************************************************
% Impostazioni di xcolor
%**************************************************************
\definecolor{webgreen}{rgb}{0,.5,0}
\definecolor{webbrown}{rgb}{.6,0,0}
\definecolor{red}{rgb}{0.6,0,0}
\definecolor{blue}{rgb}{0,0,0.6}
\definecolor{green}{rgb}{0,0.8,0}
\definecolor{cyan}{rgb}{0.0,0.6,0.6}
\definecolor{footer-gray}{HTML}{808080}
\definecolor{light-gray}{gray}{0.6} 
\definecolor{lighter-gray}{gray}{0.75} 
\definecolor{lighter-grayer}{gray}{0.85} 
\definecolor{lightest-gray}{gray}{0.94} 
\definecolor{codegreen}{rgb}{0,0.4,0.2}
\definecolor{codegray}{rgb}{0.5,0.5,0.5}
\definecolor{codepurple}{rgb}{0.58,0,0.82}
\definecolor{backcolour}{rgb}{0.95,0.95,0.96}
\newcommand{\code}[1]{\colorbox{lighter-grayer}{\texttt{#1}}}
\newcommand{\codeDark}[1]{\colorbox{black!70}{\color{white}\texttt{#1}}}


%**************************************************************
% Altro
%**************************************************************

\newcommand{\omissis}{[\dots\negthinspace]} % produce [...]

% eccezioni all'algoritmo di sillabazione
\hyphenation
{
    ma-cro-istru-zio-ne
    gi-ral-din
}

\newcommand{\sectionname}{sezione}
\addto\captionsitalian{\renewcommand{\figurename}{Figura}
                       \renewcommand{\tablename}{Tabella}}

\newcommand{\glsfirstoccur}{\ap{{[g]}}}

\newcommand{\intro}[1]{\emph{\textsf{#1}}}

%**************************************************************
% Environment per ``rischi''
%**************************************************************
\newcounter{riskcounter}                % define a counter
\setcounter{riskcounter}{0}             % set the counter to some initial value

%%%% Parameters
% #1: Title
\newenvironment{risk}[1]{
    \refstepcounter{riskcounter}        % increment counter
    \par \noindent                      % start new paragraph
    \textbf{\arabic{riskcounter}. #1}   % display the title before the 
                                        % content of the environment is displayed 
}{
    \par\medskip
}

\newcommand{\riskname}{Rischio}

\newcommand{\riskdescription}[1]{\textbf{\\Descrizione:} #1.}

\newcommand{\risksolution}[1]{\textbf{\\Soluzione:} #1.}

%**************************************************************
% Environment per ``use case''
%**************************************************************
\newcounter{usecasecounter}             % define a counter
\setcounter{usecasecounter}{0}          % set the counter to some initial value

%%%% Parameters
% #1: ID
% #2: Nome
\newenvironment{usecase}[2]{
    \renewcommand{\theusecasecounter}{\usecasename #1}  % this is where the display of 
                                                        % the counter is overwritten/modified
    \refstepcounter{usecasecounter}             % increment counter
    \vspace{10pt}
    \par \noindent                              % start new paragraph
    {\large \textbf{\usecasename #1: #2}}       % display the title before the 
                                                % content of the environment is displayed 
    \medskip
}{
    \medskip
}

\newcommand{\usecasename}{UC}

\newcommand{\usecaseactors}[1]{\textbf{\\Attori Principali:} #1. \vspace{4pt}}
\newcommand{\usecasepre}[1]{\textbf{\\Precondizioni:} #1. \vspace{4pt}}
\newcommand{\usecasedesc}[1]{\textbf{\\Descrizione:} #1. \vspace{4pt}}
\newcommand{\usecasepost}[1]{\textbf{\\Postcondizioni:} #1. \vspace{4pt}}
\newcommand{\usecasealt}[1]{\textbf{\\Scenario Alternativo:} #1. \vspace{4pt}}

%**************************************************************
% Environment per ``namespace description''
%**************************************************************

\newenvironment{namespacedesc}{
    \vspace{10pt}
    \par \noindent                              % start new paragraph
    \begin{description} 
}{
    \end{description}
    \medskip
}

\newcommand{\classdesc}[2]{\item[\textbf{#1:}] #2}

                     % file con le impostazioni personali

\begin{document}
%**************************************************************
% Materiale iniziale
%**************************************************************
\frontmatter
% !TEX encoding = UTF-8
% !TEX TS-program = pdflatex
% !TEX root = ../thesis.tex

%**************************************************************
% Frontespizio 
%**************************************************************
\begin{titlepage}

\begin{center}

\begin{LARGE}
\textbf{\myUni}\\
\end{LARGE}

\vspace{10pt}

\begin{Large}
\textsc{\myDepartment}\\
\end{Large}

\vspace{10pt}

\begin{large}
\textsc{\myFaculty}\\
\end{large}

\vspace{30pt}
\begin{figure}[htbp]
\begin{center}
\includegraphics[height=6cm]{logo-unipd}
\end{center}
\end{figure}
\vspace{1pt} 

\begin{LARGE}
\begin{center}
\textbf{\myTitle}\\
\end{center}
\end{LARGE}

\vspace{10pt} 

\begin{large}
\textsl{\myDegree}\\
\end{large}

\vspace{20pt} 

\begin{large}
\begin{flushleft}
\textit{Supervisor}\\ 
\vspace{3pt} 
\profTitle \myProf \\
\vspace{7pt} 
\textit{Co-Supervisors}\\ 
\vspace{3pt} 
Prof. Mauro Conti\\
\vspace{2pt} 
Dott. Mattia Zago
\end{flushleft}

\begin{flushright}
\textit{Graduating}\\ 
\vspace{5pt} 
\myName
\end{flushright}
\end{large}

\vspace{20pt}

\line(1, 0){338} \\
\begin{normalsize}
\textsc{Academic Year \myAA}
\end{normalsize}

\end{center}
\end{titlepage} 
\input{begin-end/colophon}
% !TEX encoding = UTF-8
% !TEX TS-program = pdflatex
% !TEX root = ../thesis.tex

%**************************************************************
% Sommario
%**************************************************************
\cleardoublepage
\phantomsection
\pdfbookmark{Summary}{Summary}
\begingroup
\let\clearpage\relax
\let\cleardoublepage\relax
\let\cleardoublepage\relax

\chapter*{Abstract}

Nowadays, most of our data is owned by private companies, and everyone knows 
everything about us because privacy online is not well preserved. Imagining a 
world different from this is difficult, but things can change thanks to 
Self-Sovereign Identity (SSI).\\
SSI's approach aims to bring credentials back to the actual owners, the people.
This is possible through cryptography and secure authentication layers 
(e.g., OAuth, OpenIDConnect).\\
The developed product embraces this philosophy and offers a solution where the 
users are the holders, issuers, or verifiers of VCs (Verifiable Credentials). 
Specifically, will be developed software agents who create, issue, verify, 
modify or even revoke the credentials, leveraging an SSI Kit.\\\\
% Ultimately, we will merge this solution with the blockchain, a sort of 
% transparent and distributed database, where we can make public what we truly want to 
% share (e.g., credentials verification by a trusted verifier).\\\\
This paper describes in detail Matteo Casonato's (approx.) three hundred hours 
of internship at the company Athesys S.r.l. (actually, working for its 
sub-startup Monokee). The goal was to merge SSI off-chain (i.e., outside the 
blockchain) operations with on-chain smart contracts.\\
In particular, the job has been divided into three macro stages:
\begin{enumerate}
    \item Deep dive into the SSI technology, studying all of its primitives 
    and problem analyzation;
    \item Development of a Software Development Kit (SDK), which enabled us 
    to dialog with an SSI Kit (off-chain logic); in the meantime, my friend 
    and co-worker Matteo Midena developed the smart contracts (on-chain logic);
    \item Merge off-chain and on-chain solutions in a proof of concept web 
    application.
\end{enumerate}

%\vfill
%
%\selectlanguage{english}
%\pdfbookmark{Abstract}{Abstract}
%\chapter*{Abstract}
%
%\selectlanguage{italian}

\endgroup			

\vfill


% !TEX encoding = UTF-8
% !TEX TS-program = pdflatex
% !TEX root = ../thesis.tex

%**************************************************************
% Ringraziamenti
%**************************************************************
\cleardoublepage
\phantomsection
\pdfbookmark{Acknowledgements}{Acknowledgements}

\begin{flushright}{
	\slshape    
	``If you always do what you’ve always done, \\
	you’ll always get what you’ve always got.''} \\ 
	\medskip
    --- Henry Ford
\end{flushright}


\bigskip

\begingroup
\let\clearpage\relax
\let\cleardoublepage\relax
\let\cleardoublepage\relax

\chapter*{Acknowledgments}

\noindent \textit{First of all, I would like to thank the people who helped 
me during the writing of this paper: my supervisor Dott. Alessandro Brighente 
and my co-supervisors Prof. Mauro Conti and Dott. Mattia Zago.}\\

\noindent \textit{Also, I want to thank my parents, who have always supported
 me, and never stopped me from doing anything I truly wanted.}\\

\noindent \textit{Finally, I thank my friends, who have eased these sometimes 
intense but very satisfying years.}\\
\bigskip

\noindent\textit{\myLocation, \myTime}
\hfill \myName

\endgroup


\input{begin-end/indexes}
\cleardoublepage

%**************************************************************
% Materiale principale
%**************************************************************
\mainmatter
% !TEX encoding = UTF-8
% !TEX TS-program = pdflatex
% !TEX root = ../thesis.tex

\chapter{Introduction}
This chapter will introduce the problem: what we are analyzing, why 
this problem exists, how it is defined, and how it can be resolved. 
Also will be introduced the company, the internship, and the work 
methodology.
% /*//////////////////////////////////////////////////////////////
%                             SECTION
% //////////////////////////////////////////////////////////////*/
\section{The problem}
As stated in the abstract, the main problem is preserving the ownership 
of people's data. In order to achieve this objective, we will pass through 
problems like interoperability, privacy safeguarding, law compliance, security,
and others.
\vspace*{0.3cm}\\
Assuming we can create a system where people hold their credentials 
(called Verifiable Credentials, or VCs):
\begin{enumerate}
    \setlength\itemsep{-0.3em}
    \item How can these credentials be shown to and verified in the 
    same manner by different actors?
    \item Can we demonstrate something without revealing it, preserving
    our privacy this way?
    \item Is it possible to save on blockchain people's data, or are we
    going against specific privacy laws?
    \item Are credentials susceptible to attacks from hackers trying to 
    steal our data?
\end{enumerate}
Thanks to Self-Sovereign Identity, we can give a positive answer to all
of these questions, but as is often the case, we have to deal with 
compromises.

\paragraph{Worldwide scenario}
The current situation is clear. Every time we interact with a new website,
we may want to interact with it, and to do so, we have to register to create 
a profile. In this phase, we have to give our data to the company, and they 
will be stored in their databases.

\paragraph{Problem identification}
Let us now try to answer the previous questions to check how the present 
context is managed:
\begin{enumerate}
    \item \textbf{Interoperability} | We could have two cases. In the first one, 
    we use a technology that enables us to use our existing account on 
    multiple websites, which integrates this solution, for example, "Sign-Up 
    with Google". Here our data is owned by Google, which shares them (if we 
    grant permission) with third parties, and no one prohibits third parties 
    from keeping our shared data saved. In the second case, we must register 
    each time if the third party does not integrate other "Sign-Up with \textbf{*}" 
    solutions. In both cases, third parties can collect our data (in the 
    first case, Google explicitly knows our interests, but there is a minimum
    degree of interoperability). Also, in most cases, companies will let us 
    create multiple accounts without verifying our data (one exception to 
    this is the use of KYC).
    \item \textbf{Privacy} | In some cases, we must show our data
    with complete transparency: for example, the police stop us on the 
    street and ask for our details. Nevertheless, let us suppose we want to 
    demonstrate something without revealing the details. For example, 
    someone has graduated and wants to demonstrate it without revealing 
    his final grade. We can do this thanks to a cryptography method 
    called \textit{Zero-Knowledge Proof}. However, this has not yet been
    implemented in most current systems.
    \item \textbf{Law compliance} | If we consider saving users' data in 
    blockchains, this problem does not exist as we examine centralized 
    systems which do not use them. By the way, of course, there are privacy 
    laws companies must follow (like GDPR).
    \item \textbf{Security} | Our information is stored in databases. With 
    a data breach, considering a centralized system, a malicious actor can 
    access all users' data at once. Sadly, this happens often. So often 
    that someone has made a website where anyone can check if his data has 
    been stolen online at least once (\href{https://haveibeenpwned.com/}
    {Have I Been Pwned?}).
\end{enumerate}

\paragraph{Problem statement}
With the above considerations, it is clear that the existing systems work 
but could be significantly improved. In fact, interoperability enhancement 
would mean privacy and security penalization. Compromises exist, but if the 
system is well designed, they can be significantly reduced or at least 
moved to less dangerous areas. Here, the need for a more secure way to store
user data arises. A way that intersects the analyzed points, bringing new 
power to people and reducing that of companies. This is the Self-Sovereign 
Identity's principle, which the developed solution will leverage.

\paragraph{Approaches} SSI's concept is pretty simple, as opposed to its 
(in development) implementation. Everyone has different relationships or 
unique sets of identifying information. This information could include 
birth date, citizenship, university degrees, or business licenses. In the 
physical world, these are represented as cards and certificates that the 
identity holder holds in their wallet or a safe place like a safety deposit 
box. They are presented when the person needs to prove their identity or 
something about it.\\
Self-sovereign identity (SSI) brings the same freedom and personal autonomy
to the internet in a safe and trustworthy identity management system. 
SSI means the individual (or organization) manages the elements that make 
up their identity, and he digitally controls access to those credentials,
called Verifiable Credentials (or VCs). They are digital representations of
information that can be verified by a third party.
\vspace*{0.3cm}\\
This is achievable by involving three participants:
\begin{enumerate}
    \item \textbf{Holder} | The holder is an individual in the scenario, 
    although it can also be an organization/company. The holder is the 
    entity that holds the credential.
    \item \textbf{Issuer} | The issuer is the institution, be it a company, 
    certifier body, or governmental organization, that has been awarded a 
    level of trust to provide information (i.e., a public body that issued 
    a passport)
    \item \textbf{Verifier} | The verifier is the individual, organization,
    company, or government with whom the holder must prove information's 
    legitimacy and trustworthiness.
\end{enumerate}
The Verifiable Data Registry grants the trust: here are stored schemas and 
identifiers (linked to the credentials) that the verifiers use to check 
data validity without the issuer's intervention.
\vspace*{0.3cm}\\
To make a preliminary check of this solution's viability, let us try to 
answer the previous four questions, considering the new scenario:
\begin{enumerate}
    \item \textbf{Interoperability} | 
    \item \textbf{Privacy} | 
    \item \textbf{Law compliance} | 
    \item \textbf{Security} | 
\end{enumerate}
Check delle 4 domande --> sono ok --> 10 principi
% /*//////////////////////////////////////////////////////////////
%                             SECTION
% //////////////////////////////////////////////////////////////*/
\section{Basic use cases}
% /*//////////////////////////////////////////////////////////////
%                             SECTION
% //////////////////////////////////////////////////////////////*/
\section{Company, internship, work methodology}

% !TEX encoding = UTF-8
% !TEX TS-program = pdflatex
% !TEX root = ../thesis.tex

\chapter{State of the art and technology background}
This chapter presents the pre-concepts needed to comprehend this paper's content fully.
As is understandable from the introduction, they are about Self-Sovereign Identity and 
blockchains. In addition, state of the art will be analyzed to see what has already been
done and what can be improved.
% /*//////////////////////////////////////////////////////////////
%                       TECHNOLOGY CONCEPTS
% //////////////////////////////////////////////////////////////*/
\section{Technology concepts}
This section will explain in detail SSI and blockchain technologies.
\subsection{Self-Sovereign Identity concepts}
Here can be read a brief reprise of what has already been saying about Self-Sovereign 
Identity and a description of its main primitives: VCs, VPs, and DIDs.
\subsubsection{Self-Sovereign Identity}
Self-Sovereign Identity is an approach to digital identity that gives individuals 
control over their data. SSI addresses the difficulty of establishing trust in 
interaction and allows people to interact in the digital world with the same freedom 
and ability to trust as they have in the offline world.
\vspace*{0.3cm}\\
To be trusted, a party in an interaction will present credentials to other parties, 
and those parties can verify that the credentials come from an \textbf{issuer} they trust.
This way, the \textbf{verifier}'s trust in the issuer is transferred to the credential 
\textbf{holder} (or \textbf{prover}). This basic structure of SSI with three participants 
is sometimes called the "triangle of trust.", simply because you need an element of trust
among these entities for them to work together.
\vspace*{0.3cm}\\
While this does not mean that there is a legal partnership or understanding between the 
entities involved, it does mean that each of the entities is willing to examine the 
credibility of the other, and this implicit trust is what constitutes this term.
\begin{center}
    \includegraphics[scale=0.2]{chapter2/triangleTrust2.jpeg}
    \captionof{figure}{The triangle of trust: Prover, Issuer, and Verifier (by Tykn)}
\end{center}
\subsubsection{Verifiable Credential (VC)}
A verifiable credential can represent all of the same information that a physical 
credential represents. The addition of technologies, such as digital signatures, 
makes verifiable credentials more tamper-evident and more trustworthy than their 
physical counterparts.\\
\begin{center}
    \vspace*{-0.5cm}
    \includegraphics[scale=0.2]{chapter2/exampleVc.png}
    \captionof{figure}{Example of verifiable credential (VC)}
\end{center}
Holders of verifiable credentials can generate verifiable presentations and then share 
these verifiable presentations with verifiers to prove they possess verifiable 
credentials with certain characteristics.\\
Both verifiable credentials and verifiable presentations can be transmitted rapidly, 
making them more convenient than their physical counterparts when trying to establish 
trust at a distance.
The three main components of a VC are:
\begin{enumerate}
    \item \textbf{Metadata}: cryptographically signed by the issuer. It describes the credential
    properties, such as the issuer, the expiry date and time, a public key to use 
    for verification purposes, the revocation mechanism, and other information;
    \item \textbf{Claims}: a statement made about a subject. Example: “Janice’s date of 
    birth is 01/01/1990.”
    \item \textbf{Proofs}: a proof is data about the identity holder that allows others 
    to verify the source of the data (i.e., the issuer), check that the data belongs to 
    (only) the holder, that the data has not been tampered with, and finally, that the 
    issuer has not revoked the data.
\end{enumerate}

\subsubsection{Verifiable Presentation (VP)}
A verifiable presentation expresses data from one or more verifiable credentials and is 
packaged in such a way that the authorship of the data is verifiable. If verifiable 
credentials are presented directly, they become verifiable presentations. Data formats 
derived from verifiable credentials that are cryptographically verifiable but do not 
themselves contain verifiable credentials might also be verifiable presentations.
\begin{center}
    \includegraphics[scale=0.18]{chapter2/exampleVp.png}
    \captionof{figure}{Example of verifiable presentation (VP)}
\end{center}
The data in a presentation is often about the same subject but might have been issued by 
multiple issuers. The aggregation of this information typically expresses an aspect of 
a person, organization, or entity.
\subsubsection{Decentralized Identifier (DID)}
\subsubsection{JSON, JWS and JWT}

\subsection{Blockchain concepts}
\subsubsection{Blockchain}
\subsubsection{Permissionless and permissioned blockchains}
\subsubsection{Bitcoin}
\subsubsection{Ethereum}
\subsubsection{Hyperledger}
\subsubsection{Hyperledger Besu}
\subsubsection{Hyperledger Fabric}

\subsection{Libraries and Stack involved}
\subsubsection{EBSI}
\subsubsection{walt.id SSI Kit}

% /*//////////////////////////////////////////////////////////////
%                        STATE OF THE ART
% //////////////////////////////////////////////////////////////*/
\section{State of the art}

% !TEX encoding = UTF-8
% !TEX TS-program = pdflatex
% !TEX root = ../thesis.tex

\chapter{Solution}
Now we discuss the path we decided to take, how we developed the software, 
and the technologies we leveraged. Then, we outline the final achievements 
and what can be done to enhance the PoC potential.
\section{Solution proposal}
After the conducted analysis in the first two weeks, we concluded that building a 
new system from zero would have needed too much time and effort and especially would 
have required specific skills we did not have.\\
First, we recall that we are building agents for verifiable credentials interaction. 
In a full stack product, our solution is placed between the users, who use secure 
communication protocols\footnote{For example, OAuth or OIDC, as can be seen in the
Figure 3.1},  and VDRs\footnote{For example EBSI, Sovrin or IBSI, blockchain used for
SSI purposes}, which store the DIDs, credentials schema, verification policies, 
and more.
\begin{center}
    \includegraphics[scale=0.28]{chapter3/problem_schema.png}
    \captionof{figure}{Ideally, Monokee will have its own DID method (did:monokee), 
    through which will be generated identifiers that will hide (at least, as far as 
    the user is concerned) the blockchain where it is located.}
\end{center}
\vspace*{0.5cm}
The final software structure has three main components:
\begin{itemize}
    \item \textbf{Frontend}: it allows the user to interact with the system's core 
    functionalities and serves as an interface for every SSI Kit SDK function.
    \item \textbf{Backend}: it is needed for security, as we will analyze 
    further, and for cryptographic functions that the frontend could not execute.
    \item \textbf{SSI Kit SDK}: it exposes all the SSI functionalities, and enables 
    the user to create keys and DIDs, issue VC, present them as VPs, and more.
    \item \textbf{Smart Contracts}: for what concerns SSI Kit integration, the 
    contracts serve as trusted verifiers and verification results register. Some 
    contracts emit ERC-721 tokens, which let the user request the diploma, but they 
    will not be discussed here.
\end{itemize}
\begin{center}
    \includegraphics[scale=0.23]{chapter3/structure.png}
    \captionof{figure}{Solution visual representation}
\end{center}

\clearpage
\section{Solution development}
In the following sections, we discuss the technologies we used to build the
solution, and we explain the main functionalities of the system.
\subsection{Technologies and Tools}
Before explaining the solution, we list the languages and tools we leveraged to develop it,
for both SSI Kit SDK and PoC.

\subsubsection{Common tools and languages}
\begin{itemize}
    \setlength\itemsep{-0.1em}
    \item \texttt{Typescript}: a strongly typed programming language that builds 
    on JavaScript. It was chosen because the other Monokee's modules were in 
    Typescript, so it would have been easier for the team to integrate. Also, it is 
    very convenient for its strongly typed nature;
    \item \texttt{JSON}: as already covered in \hyperref[subsubsec:json]{Chapter 2}, 
    JSON is a lightweight data-interchange format. It has been extensively used,
    mainly for credentials representation and API calls;
    \item \texttt{Node.js}: a JavaScript runtime built on Chrome's V8 JavaScript
    engine. It has been used for code execution;
    \item \texttt{npm}: a package manager for the JavaScript programming language.
    It has been used to manage the dependencies of the projects;
    \item \texttt{Git}: a free and open-source distributed version control system
    used for tracking and collaboration purposes;
    \item \texttt{Visual Studio Code}: the Integrated Development Environment (IDE)
    we have used for the solution development.
    
\end{itemize}

\subsubsection[SSI Kit SDK]{SSI Kit SDK\footnote{\texttt{uuid}, \texttt{rfc4648}, \texttt{sha256}, 
and \texttt{nacl} have been used just to generate tokens used for credentials 
revocation, as will be further explained}}

\begin{itemize}
    \setlength\itemsep{-0.1em}
    \item \texttt{jest}: a JavaScript testing framework. It has been used to test
    the SSI Kit SDK components;
    \item \texttt{waltid-ssikit}: the library written in Kotlin/Java that provides 
    the SSI functionalities set. The developed SDK is a Typescript wrapper of this 
    library;
    \item \texttt{axios}: a promise-based HTTP client for the browser and node.js,
    used to make API calls to waltid-ssikit;
    \item \texttt{uuid}: a library used to generate RFC-compliant Universally Unique
    Identifiers (UUIDs);
    \item \texttt{rfc4648}: a library used to encode and decode data in Base32 format;
    \item \texttt{sha256}: a library used to generate SHA-256 hashes;
    \item \texttt{nacl}: a library used to decode UTF8 strings;
\end{itemize}

\subsubsection{Frontend}
\begin{itemize}
    \setlength\itemsep{-0.1em}
    \item \texttt{React.js}: a JavaScript library for building user interfaces. It
    has been used to build the frontend;
    \item \texttt{Chakra-UI}: a simple, modular and accessible components library,
    used with \texttt{React.js} to build the frontend.
    \item \texttt{ethers}: a library used to interact with Ethereum Virtual
    Machine compatible blockchains;
    \item \texttt{wagmi}: a collection of React Hooks containing everything needed
    to start working with Ethereum; it has been used to interact with the smart
    contracts;
    \item \texttt{RainbowKit}: RainbowKit is a React library that makes it easy to 
    add the wallet connection, e.g., for Metamask integration.
    \item \texttt{GraphQL}: the query language used by The Graph;
    \item \texttt{ssikit-sdk}: the developed Typescript SDK used to interact with 
    the SSI Kit library.
    \item \texttt{The Graph}: a decentralized protocol for indexing and querying
    data from blockchains, starting with Ethereum. It makes it possible to query 
    data that is difficult to query directly. It has been used to query The
    deployed smart contracts.
    \item \texttt{smart contracts suite}: a collection of smart contracts used to
    register the verifications and the verification results on-chain.
\end{itemize}

\subsubsection{Backend}
\begin{itemize}
    \setlength\itemsep{-0.1em}
    \item \texttt{Express.js}: a web application framework for Node.js. It has been
    used to build the backend, where the cryptographic functions are executed; the frontend
    calls them through API calls;
    \item \texttt{ssikit-sdk}: the developed Typescript SDK used to interact with
    the SSI Kit library.
    \item \texttt{jose}: a library used to encode and decode JSON Web Tokens (JWTs),
    which have been used to represent private and public keys;
    \item \texttt{nodemon}: a tool that automatically restarts the node application
    when file changes in the directory are detected.
\end{itemize}

\subsection{SSI Kit SDK development}
\subsection{Smart Contract integrations - Web App Proof of Concept}

\section{Discussion}
\subsection{Achievements}
\subsection{Acquired knowledge}
\subsection{Future developments}
\subsection{Personal evaluation}


%**************************************************************
% Materiale finale
%**************************************************************
\backmatter
\chapter*{Conclusion}
Taking up what was said in the abstract, we can confirm that the final product embraces 
SSI concepts and tries to take it to the next level with the help of smart contracts.  
The developed SDK enables the issuers to release Verifiable Credentials to holders who 
own them and present them to verifiers who can confirm their validity. Thanks to smart 
contracts, we can register on-chain verification results to make them public and speed 
up the following verifications. The final proof of concept proves that off-chain SSI 
primitives can be reflected on-chain. To do so, compromises are needed to preserve 
privacy (e.g., using a permissioned blockchain as VDR simplifies things).
\vspace{0.3cm}\\
Our final product leaves room for numerous additional features, meaning that this was 
not thought of as definitive software but as a beginning for the following implementations.
We are confident that Self-Sovereign Identity will catch on sooner or later, and the 
conclusive result of this thesis offers just a taste of what these innovative and
promising technologies could bring.
\printglossary[type=\acronymtype, title=Acronimi e abbreviazioni, toctitle=Acronimi e abbreviazioni]
\printglossary[type=main, title=Glossary, toctitle=Glossary]
% !TEX encoding = UTF-8
% !TEX TS-program = pdflatex
% !TEX root = ../thesis.tex

%**************************************************************
% Bibliografia
%**************************************************************
\cleardoublepage
\chapter{Bibliography}
\nocite{*}

% Stampa i riferimenti bibliografici
\printbibliography[
    heading=subbibliography,
    title={Bibliographical references},
    type=article,
]

% Stampa i siti web consultati
\printbibliography[
    heading=subbibliography,
    title={Websites consulted},
    type=online,
]

\end{document}
