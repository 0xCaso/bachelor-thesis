%**************************************************************
% file contenente le impostazioni della thesis
%**************************************************************

%**************************************************************
% Frontespizio
%**************************************************************

% Autore
\newcommand{\myName}{Matteo Casonato}                                    
\newcommand{\myTitle}{Owning your data through Self-Sovereign Identity: agents implementation for Verifiable Credentials interaction}

% Tipo di thesis                   
\newcommand{\myDegree}{Bachelor thesis}

% Università             
\newcommand{\myUni}{University of Padua}

% Facoltà       
\newcommand{\myFaculty}{Bachelor's Degree in Computer Science}

% Dipartimento
\newcommand{\myDepartment}{Department of Mathematics "Tullio Levi-Civita"}

% Titolo del relatore
\newcommand{\profTitle}{Prof. }

% Relatore
\newcommand{\myProf}{Alessandro Brighente}

% Luogo
\newcommand{\myLocation}{Padova}

% Anno accademico
\newcommand{\myAA}{2021-2022}

% Data discussione
\newcommand{\myTime}{September 2022}


%**************************************************************
% Impostazioni di impaginazione
% see: http://wwwcdf.pd.infn.it/AppuntiLinux/a2547.htm
%**************************************************************

\setlength{\parindent}{14pt}   % larghezza rientro della prima riga
\setlength{\parskip}{0pt}   % distanza tra i paragrafi

% PER STAMPARE TESI, COMMENTARE LE 4 RIGHE SOTTO
% E SETTARE TWOSIDE IN DOCUMENTCLASS

\pdfpageheight=\paperheight
\pdfpagewidth=\paperwidth
\setlength\oddsidemargin{\dimexpr(\paperwidth-\textwidth)/2 - 1in\relax}
\setlength\evensidemargin{\oddsidemargin}

%**************************************************************
% Impostazioni di biblatex
%**************************************************************
\bibliography{bibliography} % database di biblatex 

\defbibheading{bibliography} {
    \cleardoublepage
    \phantomsection 
    \addcontentsline{toc}{chapter}{\bibname}
    \chapter*{\bibname\markboth{\bibname}{\bibname}}
}

\setlength\bibitemsep{1.5\itemsep} % spazio tra entry
\setlength{\skip\footins}{1.2pc plus 10pt minus 2pt} % spazio tra entry e footnote

\DeclareBibliographyCategory{opere}
\DeclareBibliographyCategory{web}
\DeclareBibliographyDriver{online}{%
  \usebibmacro{bibindex}%
  \usebibmacro{begentry}%
  \usebibmacro{author/editor+others/translator+others}%
  \setunit{\adddot\addspace}%
    \iffieldundef{year}
        {}
        {\printtext[parens]{\usebibmacro{date}}}
  \setunit{\adddot\addspace}%
  \usebibmacro{title}%
  \setunit{\adddot\addspace}%
  \printlist{language}%
  \setunit{\adddot\addspace}%
  \usebibmacro{byauthor}%
  \setunit{\adddot\addspace}%
  \usebibmacro{byeditor+others}%
  \setunit{\adddot\addspace}%
  \printfield{version}%
  \setunit{\adddot\addspace}%
  \printfield{note}%
  \newunit\newblock
  \printlist{organization}%
  \setunit{\adddot\addspace}%
  \iftoggle{bbx:eprint}
    {\usebibmacro{eprint}}
    {}%
  \newunit\newblock
  \usebibmacro{url+urldate}%
  \setunit{\adddot\addspace}%
  \usebibmacro{addendum+pubstate}%
  \setunit{\bibpagerefpunct}\newblock
  \usebibmacro{pageref}%
  \newunit\newblock
  \usebibmacro{related}%
  \usebibmacro{finentry}%
}

\addtocategory{opere}{article:zkkyc}
\addtocategory{opere}{article:soulboundpaper}
\addtocategory{opere}{article:soulbound}
\addtocategory{web}{site:agile-manifesto}

\defbibheading{opere}{\section*{Bibliographical references}}
\defbibheading{web}{\section*{Websites consulted}}
\defbibheading{booklets}{\section*{Booklets / Presentations}}


%**************************************************************
% Impostazioni di caption
%**************************************************************
\captionsetup{
    tableposition=top,
    figureposition=bottom,
    font=small,
    format=hang,
    labelfont=bf
}

%**************************************************************
% Impostazioni di glossaries
%**************************************************************
\makeglossaries
%**************************************************************
% Acronimi
%**************************************************************
\newacronym{api}{API}{Application Programming Interface}
\newacronym{cli}{CLI}{Command Line Interface}
\newacronym{crud}{CRUD}{Create-Read-Update-Delete}
\newacronym{did}{DID}{Decentralized Identifier}
\newacronym{ebsi}{EBSI}{European Blockchain Services Infrastructure}
\newacronym{erc}{ERC}{Ethereum Request for Comments}
\newacronym{evm}{EVM}{Ethereum Virtual Machine}
\newacronym{gdpr}{GDPR}{General Data Protection Regulation}
\newacronym{json}{JSON}{JavaScript Object Notation}
\newacronym{jvm}{JVM}{Java Virtual Machine}
\newacronym{jwk}{JWK}{JSON Web Key}
\newacronym{jws}{JWS}{JSON Web Signature}
\newacronym{jwt}{JWT}{JSON Web Token}
\newacronym{kyc}{KYC}{Know Your Customer}
\newacronym{nft}{NFT}{Non-Fungible Token}
\newacronym{oidc}{OIDC}{OpenID Connect}
\newacronym{pem}{PEM}{Privacy Enhanced Mail}
\newacronym{poc}{POC}{Proof of Concept}
\newacronym{rest}{REST}{Representational State Transfer}
\newacronym{rfc}{RFC}{Request for Comments}
\newacronym{ui}{UI}{User Interface}
\newacronym{uuid}{UUID}{Universally Unique IDentifier}
\newacronym{ux}{UX}{User Experience}
\newacronym{vc}{VC}{Verifiable Credential}
\newacronym{vp}{VP}{Verifiable Presentation}
\newacronym{vdr}{VDR}{Verifiable Data Registry}
\newacronym{sdk}{SDK}{Software Development Kit}
\newacronym{ssi}{SSI}{Self-Sovereign Identity}
\newacronym{uri}{URI}{Uniform Resource Identifier}
\newacronym{w3c}{W3C}{World Wide Web Consortium}
\newacronym{zkp}{ZKP}{Zero-Knowledge Proof}
\newacronym{zksnark}{ZK-SNARK}{Zero-Knowledge Succinct Non-interactive ARguments of Knowledge}
 % database di termini


%**************************************************************
% Impostazioni di graphicx
%**************************************************************
\graphicspath{{img/}} % cartella dove sono riposte le immagini


%**************************************************************
% Impostazioni di hyperref
%**************************************************************
\hypersetup{
    %hyperfootnotes=false,
    %pdfpagelabels,
    %draft,	% = elimina tutti i link (utile per stampe in bianco e nero)
    colorlinks=true,
    linktocpage=true,
    pdfstartpage=1,
    pdfstartview=,
    % decommenta la riga seguente per avere link in nero (per esempio per la stampa in bianco e nero)
    %colorlinks=false, linktocpage=false, pdfborder={0 0 0}, pdfstartpage=1, pdfstartview=FitV,
    breaklinks=true,
    pdfpagemode=UseNone,
    pageanchor=true,
    pdfpagemode=UseOutlines,
    plainpages=false,
    bookmarksnumbered,
    bookmarksopen=true,
    bookmarksopenlevel=1,
    hypertexnames=true,
    pdfhighlight=/O,
    %nesting=true,
    %frenchlinks,
    urlcolor=webbrown,
    linkcolor=RoyalBlue,
    citecolor=webgreen,
    %pagecolor=RoyalBlue,
    %urlcolor=Black, linkcolor=Black, citecolor=Black, %pagecolor=Black,
    pdftitle={\myTitle},
    pdfauthor={\textcopyright\ \myName, \myUni, \myFaculty},
    pdfsubject={},
    pdfkeywords={},
    pdfcreator={pdfLaTeX},
    pdfproducer={LaTeX}
}

%**************************************************************
% Impostazioni di itemize
%**************************************************************
\renewcommand{\labelitemi}{$\ast$}
% \renewcommand{\labelitemi}{$\bullet$}
% \renewcommand{\labelitemii}{$\cdot$}
% \renewcommand{\labelitemiii}{$\diamond$}
%\renewcommand{\labelitemiv}{$\ast$}


%**************************************************************
% Impostazioni di listings
%**************************************************************
\lstset{
    language=[LaTeX]Tex,%C++,
    keywordstyle=\color{RoyalBlue}, %\bfseries,
    basicstyle=\small\ttfamily,
    %identifierstyle=\color{NavyBlue},
    commentstyle=\color{Green}\ttfamily,
    stringstyle=\rmfamily,
    numbers=none, %left,%
    numberstyle=\scriptsize, %\tiny
    stepnumber=5,
    numbersep=8pt,
    showstringspaces=false,
    breaklines=true,
    frameround=ftff,
    frame=single
} 


%**************************************************************
% Impostazioni di xcolor
%**************************************************************
\definecolor{webgreen}{rgb}{0,.5,0}
\definecolor{webbrown}{rgb}{.6,0,0}
\definecolor{red}{rgb}{0.6,0,0}
\definecolor{blue}{rgb}{0,0,0.6}
\definecolor{green}{rgb}{0,0.8,0}
\definecolor{cyan}{rgb}{0.0,0.6,0.6}
\definecolor{footer-gray}{HTML}{808080}
\definecolor{light-gray}{gray}{0.6} 
\definecolor{lighter-gray}{gray}{0.75} 
\definecolor{lighter-grayer}{gray}{0.85} 
\definecolor{lightest-gray}{gray}{0.94} 
\definecolor{codegreen}{rgb}{0,0.4,0.2}
\definecolor{codegray}{rgb}{0.5,0.5,0.5}
\definecolor{codepurple}{rgb}{0.58,0,0.82}
\definecolor{backcolour}{rgb}{0.95,0.95,0.96}
\newcommand{\code}[1]{\colorbox{lighter-grayer}{\texttt{#1}}}
\newcommand{\codeDark}[1]{\colorbox{black!70}{\color{white}\texttt{#1}}}


%**************************************************************
% Altro
%**************************************************************

\newcommand{\omissis}{[\dots\negthinspace]} % produce [...]

% eccezioni all'algoritmo di sillabazione
\hyphenation
{
    ma-cro-istru-zio-ne
    gi-ral-din
}

\newcommand{\sectionname}{sezione}
\addto\captionsitalian{\renewcommand{\figurename}{Figura}
                       \renewcommand{\tablename}{Tabella}}

\newcommand{\glsfirstoccur}{\ap{{[g]}}}

\newcommand{\intro}[1]{\emph{\textsf{#1}}}

%**************************************************************
% Environment per ``rischi''
%**************************************************************
\newcounter{riskcounter}                % define a counter
\setcounter{riskcounter}{0}             % set the counter to some initial value

%%%% Parameters
% #1: Title
\newenvironment{risk}[1]{
    \refstepcounter{riskcounter}        % increment counter
    \par \noindent                      % start new paragraph
    \textbf{\arabic{riskcounter}. #1}   % display the title before the 
                                        % content of the environment is displayed 
}{
    \par\medskip
}

\newcommand{\riskname}{Rischio}

\newcommand{\riskdescription}[1]{\textbf{\\Descrizione:} #1.}

\newcommand{\risksolution}[1]{\textbf{\\Soluzione:} #1.}

%**************************************************************
% Environment per ``use case''
%**************************************************************
\newcounter{usecasecounter}             % define a counter
\setcounter{usecasecounter}{0}          % set the counter to some initial value

%%%% Parameters
% #1: ID
% #2: Nome
\newenvironment{usecase}[2]{
    \renewcommand{\theusecasecounter}{\usecasename #1}  % this is where the display of 
                                                        % the counter is overwritten/modified
    \refstepcounter{usecasecounter}             % increment counter
    \vspace{10pt}
    \par \noindent                              % start new paragraph
    {\large \textbf{\usecasename #1: #2}}       % display the title before the 
                                                % content of the environment is displayed 
    \medskip
}{
    \medskip
}

\newcommand{\usecasename}{UC}

\newcommand{\usecaseactors}[1]{\textbf{\\Attori Principali:} #1. \vspace{4pt}}
\newcommand{\usecasepre}[1]{\textbf{\\Precondizioni:} #1. \vspace{4pt}}
\newcommand{\usecasedesc}[1]{\textbf{\\Descrizione:} #1. \vspace{4pt}}
\newcommand{\usecasepost}[1]{\textbf{\\Postcondizioni:} #1. \vspace{4pt}}
\newcommand{\usecasealt}[1]{\textbf{\\Scenario Alternativo:} #1. \vspace{4pt}}

%**************************************************************
% Environment per ``namespace description''
%**************************************************************

\newenvironment{namespacedesc}{
    \vspace{10pt}
    \par \noindent                              % start new paragraph
    \begin{description} 
}{
    \end{description}
    \medskip
}

\newcommand{\classdesc}[2]{\item[\textbf{#1:}] #2}

